%\documentstyle[10pt,twoside]{article}
\documentclass[twoside, 11pt]{article}
\setlength{\oddsidemargin}{0.25 in}
\setlength{\evensidemargin}{-0.25 in}
\setlength{\topmargin}{-0.6 in}
\setlength{\textwidth}{6.5 in}
\setlength{\textheight}{8.5 in}
\setlength{\headsep}{0.75 in}
\setlength{\parindent}{0 in}
\setlength{\parskip}{0.1 in}

\usepackage{hyperref}
\usepackage{amsmath}
\usepackage{amsfonts}
\usepackage{amssymb}
\usepackage{amsthm}
\usepackage{mathtools}
%
% The following commands sets up the lecnum (lecture number)
% counter and make various numbering schemes work relative
% to the lecture number.
%
\newcounter{lecnum}
\renewcommand{\thepage}{\thelecnum-\arabic{page}}
\renewcommand{\thesection}{\thelecnum.\arabic{section}}
\renewcommand{\theequation}{\thelecnum.\arabic{equation}}
\renewcommand{\thefigure}{\thelecnum.\arabic{figure}}
\renewcommand{\thetable}{\thelecnum.\arabic{table}}

%
% The following macro is used to generate the header.
%
\newcommand{\problemset}[3]{
   \pagestyle{myheadings}
   \thispagestyle{plain}
   \newpage
   \setcounter{lecnum}{#1}
   \setcounter{page}{1}
   \noindent
   \begin{center}
   \framebox{
      \vbox{\vspace{2mm}
    \hbox to 6.28in { {\bf COL~352~~~Introduction to Automata and Theory of Computation
                        \hfill Holi 2022} }
       \vspace{4mm}
       \hbox to 6.28in { {\Large \hfill Problem Set #1  \hfill} }
       \vspace{2mm}
       \hbox to 6.28in { {\it Out: #2 \hfill Due: #3} }
      \vspace{2mm}}
   }
   \end{center}
   \vspace*{4mm}
}

%
% Convention for citations is authors' initials followed by the year.
% For example, to cite a paper by Leighton and Maggs you would type
% \cite{LM89}, and to cite a paper by Strassen you would type \cite{S69}.
% (To avoid bibliography problems, for now we redefine the \cite command.)
%
\renewcommand{\cite}[1]{[#1]}

\input{epsf}

%Use this command for a figure; it puts a figure in wherever you want it.
%usage: \fig{NUMBER}{FIGURE-SIZE}{CAPTION}{FILENAME}
\newcommand{\fig}[4]{
            \vspace{0.2 in}
            \setlength{\epsfxsize}{#2}
            \centerline{\epsfbox{#4}}
            \begin{center}
            Figure \thelecnum.#1:~#3
            \end{center}
    }

% Use these for theorems, lemmas, proofs, etc.
\newtheorem{theorem}{Theorem}[lecnum]
\newtheorem{lemma}[theorem]{Lemma}
\newtheorem{proposition}[theorem]{Proposition}
\newtheorem{claim}[theorem]{Claim}
\newtheorem{corollary}[theorem]{Corollary}
\newtheorem{definition}[theorem]{Definition}
\newenvironment{Proof}{{\bf Proof:}}{\hfill\rule{2mm}{2mm}}

% Some useful equation alignment commands, borrowed from TeX
\makeatletter
\def\eqalign#1{\,\vcenter{\openup\jot\m@th
  \ialign{\strut\hfil$\displaystyle{##}$&$\displaystyle{{}##}$\hfil
      \crcr#1\crcr}}\,}
\def\eqalignno#1{\displ@y \tabskip\@centering
  \halign to\displaywidth{\hfil$\displaystyle{##}$\tabskip\z@skip
    &$\displaystyle{{}##}$\hfil\tabskip\@centering
    &\llap{$##$}\tabskip\z@skip\crcr
    #1\crcr}}
\def\leqalignno#1{\displ@y \tabskip\@centering
  \halign to\displaywidth{\hfil$\displaystyle{##}$\tabskip\z@skip
    &$\displaystyle{{}##}$\hfil\tabskip\@centering
    &\kern-\displaywidth\rlap{$##$}\tabskip\displaywidth\crcr
    #1\crcr}}
\makeatother

% **** IF YOU WANT TO DEFINE ADDITIONAL MACROS FOR YOURSELF, PUT THEM HERE:



\begin{document}
\problemset{3}{March 9}{{\bf March 17}}

Please attempt all problems. Read the instructions for problem sets posted on the announcement channel in MS Teams (and also \href{https://moodle.iitd.ac.in/mod/forum/discuss.php?d=23662}{here}) carefully.  Turn in your solutions via \href{https://www.gradescope.com/courses/344938}{Gradescope} by 11 pm on the due date. 


\begin{enumerate}

\item We say that a context-free grammar G is self-referential if for some non-terminal symbol $X$ we have $X \to^* \alpha X \beta$, where $\alpha, \beta \neq \varepsilon$. Show that a CFG that is not self-referential is regular.

\item Prove that the class of context-free languages is closed under intersection with regular languages. That is, prove that if $L_1$ is a context-free language and $L_2$ is a regular language, then $L_1 \cap L_2$
is a context-free language. Do this by starting with a DF

\item ({\bf 2 points}) Given two languages $L, L'$, denote by 
$$L||L' := \{x_1y_1x_2y_2 \dots x_ny_n \mid x_1x_2 \dots x_n \in L, y_1y_2\dots y_n \in L'\}$$. 

Show that if $L$ is a CFL and $L'$ is regular, then $L || L'$ is a CFL by constructing a PDA for $L || L'$. Is $L || L'$ a CFL if both $L$ and $L'$ are CFLs? Justify your answer.

\item For $A \subseteq \Sigma^*$, define 
$$cycle(A) = \{yx \mid xy \in A\}$$
For example if $A = \{aaabc\}$, then 
$$cycle(A) = \{aaabc, aabca, abcaa, bcaaa, caaab\}$$.
Show that if $A$ is a CFL then so is $cycle(A)$

\item Let $$A = \{wtw^R\mid w,t, \in \{0,1\}^* \text{ \ and \ } |w| = |t|\}$$. Show that $A$ is not a CFL. 


\item Prove the following stronger version of pumping lemma for CFLs: 
If $A$ is a CFL, then there is a number $k$ where if $s$ is any string in $A$ of length at least $k$ then $s$ may be divided into five pieces $s = uvxyz$, satisfying the conditions:
\begin{itemize}
    \item for each $i\geq 0$, $uv^ixy^iz \in A$
    \item $v \neq \varepsilon$, and $y \neq \varepsilon$, and
    \item $|vxy| \leq k$.
\end{itemize}

\item Give an example of a language that is not a CFL but nevertheless acts like a CFL in the pumping lemma for CFL (Recall we saw such an example in class while studying pumping lemma for regular languages). 


\end{enumerate}


\end{document}