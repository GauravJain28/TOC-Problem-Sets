%\documentstyle[10pt,twoside]{article}
\documentclass[twoside, 11pt]{article}
\setlength{\oddsidemargin}{0.25 in}
\setlength{\evensidemargin}{-0.25 in}
\setlength{\topmargin}{-0.6 in}
\setlength{\textwidth}{6.5 in}
\setlength{\textheight}{8.5 in}
\setlength{\headsep}{0.75 in}
\setlength{\parindent}{0 in}
\setlength{\parskip}{0.1 in}

\usepackage{hyperref}
\usepackage{amsmath}
\usepackage{amsfonts}
\usepackage{amssymb}
\usepackage{amsthm}
\usepackage{mathtools}
%
% The following commands sets up the lecnum (lecture number)
% counter and make various numbering schemes work relative
% to the lecture number.
%
\newcounter{lecnum}
\renewcommand{\thepage}{\thelecnum-\arabic{page}}
\renewcommand{\thesection}{\thelecnum.\arabic{section}}
\renewcommand{\theequation}{\thelecnum.\arabic{equation}}
\renewcommand{\thefigure}{\thelecnum.\arabic{figure}}
\renewcommand{\thetable}{\thelecnum.\arabic{table}}

%
% The following macro is used to generate the header.
%
\newcommand{\problemset}[3]{
	\pagestyle{myheadings}
	\thispagestyle{plain}
	\newpage
	\setcounter{lecnum}{#1}
	\setcounter{page}{1}
	\noindent
	\begin{center}
		\framebox{
			\vbox{\vspace{2mm}
				\hbox to 6.28in { {\bf COL~352~~~Introduction to Automata and Theory of Computation
						\hfill Holi 2022} }
				\vspace{4mm}
				\hbox to 6.28in { {\Large \hfill Problem Set #1  \hfill} }
				\vspace{2mm}
				\hbox to 6.28in { {\it Out: #2 \hfill Due: #3} }
				\vspace{2mm}}
		}
	\end{center}
	\vspace*{4mm}
}

%
% Convention for citations is authors' initials followed by the year.
% For example, to cite a paper by Leighton and Maggs you would type
% \cite{LM89}, and to cite a paper by Strassen you would type \cite{S69}.
% (To avoid bibliography problems, for now we redefine the \cite command.)
%
\renewcommand{\cite}[1]{[#1]}

\input{epsf}

%Use this command for a figure; it puts a figure in wherever you want it.
%usage: \fig{NUMBER}{FIGURE-SIZE}{CAPTION}{FILENAME}
\newcommand{\fig}[4]{
	\vspace{0.2 in}
	\setlength{\epsfxsize}{#2}
	\centerline{\epsfbox{#4}}
	\begin{center}
		Figure \thelecnum.#1:~#3
	\end{center}
}

% Use these for theorems, lemmas, proofs, etc.
\newtheorem{theorem}{Theorem}[lecnum]
\newtheorem{lemma}[theorem]{Lemma}
\newtheorem{proposition}[theorem]{Proposition}
\newtheorem{claim}[theorem]{Claim}
\newtheorem{corollary}[theorem]{Corollary}
\newtheorem{definition}[theorem]{Definition}
\newenvironment{Proof}{{\bf Proof:}}{\hfill\rule{2mm}{2mm}}

% Some useful equation alignment commands, borrowed from TeX
\makeatletter
\def\eqalign#1{\,\vcenter{\openup\jot\m@th
		\ialign{\strut\hfil$\displaystyle{##}$&$\displaystyle{{}##}$\hfil
			\crcr#1\crcr}}\,}
\def\eqalignno#1{\displ@y \tabskip\@centering
	\halign to\displaywidth{\hfil$\displaystyle{##}$\tabskip\z@skip
		&$\displaystyle{{}##}$\hfil\tabskip\@centering
		&\llap{$##$}\tabskip\z@skip\crcr
		#1\crcr}}
\def\leqalignno#1{\displ@y \tabskip\@centering
	\halign to\displaywidth{\hfil$\displaystyle{##}$\tabskip\z@skip
		&$\displaystyle{{}##}$\hfil\tabskip\@centering
		&\kern-\displaywidth\rlap{$##$}\tabskip\displaywidth\crcr
		#1\crcr}}
\makeatother

% **** IF YOU WANT TO DEFINE ADDITIONAL MACROS FOR YOURSELF, PUT THEM HERE:



\begin{document}
	\problemset{2}{February 18}{{\bf February 26}}
	
	Please attempt all problems. Read the instructions for problem sets posted on the announcement channel in MS Teams (and also \href{https://moodle.iitd.ac.in/mod/forum/discuss.php?d=23662}{here}) carefully.  Turn in your solutions via \href{https://www.gradescope.com/courses/344938}{Gradescope} by 11 pm on the due date. 
	
	
	\begin{enumerate}
		\item Prove that $L_1 = \{bin(p): p \mbox{ \ is a prime number}\}$ is not a regular language.
		\item The $n$-th Fibonacci number is defined as $F_1 = 1, F_2 = 1$, and for all $n \geq 3, F_n = F_{n-1} + F_{n-2}$. Consider the language over $\Sigma = \{a\}$
		$$ L_2 = \{a^m \mid m = F_n\} $$ Is $L_2$ regular? Justify your answer.
		
		\item If $A$ is any language, let $A_{\frac{1}{2}-}$ denote the set of all first halves of strings in $A$ so that 
		$$A_{\frac{1}{2}-} = \{x \mid \mbox{for some $y$, $|x| = |y|$ and $xy \in A$}\}$$
		Show that if $A$ is regular, then so is $A_{\frac{1}{2}-}$.
		
		\item If $A$ is any language, let $A_{\frac{1}{3}-\frac{1}{3}}$ denote the set of strings in $A$ with the middle-third removed so that 
		$$A_{\frac{1}{3}-\frac{1}{3}} = \{xz \mid \mbox{for some $y$, $|x| = |y| = |z|$ and $xyz \in A$}\}$$
		Show that if $A$ is regular, then $A_{\frac{1}{3}-\frac{1}{3}}$ is not necessarily regular.
		
		\item A 2-NFA $A$ is a 5-tuple $A = (Q, S, t, F, \Delta)$ where $Q$ is the set of states, $S$ the set of start states, $t$ is an accept state, the transition function 
		$$ \Delta: Q \times (\Sigma \cup \{\#, \$\}) \to 2^{Q \times (\{L,R\})}$$
		
		Assume that whenever $M$ accepts, it does so my moving the head (pointer) all the way to the right endmarker $\$$ and entering accept state $t$. In the subsequent two questions, we will try to prove that $2$-NFAs accept only regular languages. 
		
		\item[(a)] Let $x = a_1 \dots a_n \in \Sigma^*, a_i \in \Sigma$, $1 \leq i \leq n$. Let $a_0 = \#, a_{n+1} = \$$. Argue that $x$ is not accepted by $A$ if and only if there exist sets $W_i \subseteq Q, 0 \leq i \leq n+1$ such that the following hold:
		\begin{itemize}
			\item $S \subseteq W_0$
			\item If $u \in W_i$, $0 \leq i \leq n$, and $(v, R) \in \Delta(u, a_i)$, then $v \in W_{i+1}$
			\item If $u \in W_i$, $1 \leq i \leq n+1$, and $(v, L) \in \Delta(u, a_i)$, then $v \in W_{i-1}$ and
			\item $t \notin W_{n+1}$.
		\end{itemize}
		
		\item[(b)] Using the previous part, show that $L(A)$ is regular.
		
		\item Let $M = (Q, \Sigma, q_0, \delta, F)$ be a DFA and let $h$ be a state of $M$ called its ``home”. A synchronizing sequence for $M$ and $h$ is a string $s \in \Sigma^*$ where $\hat{\delta}(q, s) = h$ for
		every $q \in Q$. Say that $M$ is synchronizable if it has a synchronizing sequence for some state $h$. Prove that if $M$ is a $k$-state synchronizable DFA, then it has a synchronizing sequence of length at most $k^3$. Can you improve upon this bound?
		
		\item For every string $x \in \{0,1\}^+$ consider the number
		$$ 0.x = x[1] \cdot \frac{1}{2} + x[2] \cdot \frac{1}{2^2} +  \dots + x[|x|] \cdot \frac{1}{2^{|x|}}$$
		where $|x|$ is the length of $x$. For a real number $\theta \in [0,1]$ let 
		$$L_{\theta} = \{x : 0.x \leq \theta\}$$
		Prove that $L_{\theta}$ is regular if and only if $\theta$ is rational.
	\end{enumerate}
	
	
\end{document}